% Created 2017-08-31 qui 16:16
% Intended LaTeX compiler: pdflatex
\documentclass[11pt]{article}
\usepackage[latin1]{inputenc}
\usepackage[T1]{fontenc}
\usepackage{graphicx}
\usepackage{grffile}
\usepackage{longtable}
\usepackage{wrapfig}
\usepackage{rotating}
\usepackage[normalem]{ulem}
\usepackage{amsmath}
\usepackage{textcomp}
\usepackage{amssymb}
\usepackage{capt-of}
\usepackage{hyperref}
\date{\today}
\title{}
\hypersetup{
 pdfauthor={},
 pdftitle={},
 pdfkeywords={},
 pdfsubject={},
 pdfcreator={Emacs 25.2.1 (Org mode 9.0.9)}, 
 pdflang={English}}
\begin{document}

\textbf{Global Markets.} European equities are broadly down, and so is the
S\&P future, that is falling by 0.4\%. In the meantime, the Mexican peso is
selling off  against the USD, and the yield of the 10 years
Treasury is about flat at 2.24\%. Commodities: Oil is up by 0.4\% to
USD49.3.

\textbf{Local News}. Primary target drama continues, as underscored by this
morning's news. The \textbf{idea of increasing households income tax has been
rubuked} the Speaker of the Lower House and, aftewards, by the
government as well. Consequently, as highlighted by the O Globo, the
focus has been shifted to other taxes, and, at least for now,
the prime candidates seem to be taxes on financial instruments, today
exempted from it, such as, on the fixed income side, LCI, LCA and, on dividends.

Be that as it may, the truth is that decisions on taxes outgh to be
reached by August 31st, at latest. This is because, otherwise, they
won't be able to be included in the 2018's budget.

On the reform's camp, the social security is certainly coming back to
the fore. The O Globo reports the ongoing debate within the government
as of to fine-tune or not the report discussed till past
May. According to sources, two points are not negotiable: 1) minium
retirement age, and 2) equelization of retirement rights across public
and private sectors alike.

While this debate goes on, the lower clergy (centrao) is flirting with
the idea of not voting the subject this year. Down underneath, it is
calling for more room within Temer's government in order to be
persuaded to push the reform forward in the near-term.

Finally, there is some little noise with regards to TLP provisional
measure. As per O Estado de São Paulo, there are pressures to, to some
extent, to allow small firms tap into BNDES' loans at rates below
TLP. However, the rapporteour of the matter has confirmed it as of
yet.


\textbf{Agenda - Highlights}: \uline{Brazil}: CPI, July. \uline{US}:


\rowcolors{2}{grey!15}{white}
\vspace{-0.5cm}
\begin{center}
\begin{tabular}{rlllll}
\textbf{Time} & \textbf{Country} & \textbf{Indicator} & \textbf{Period} & \textbf{Forecast} & \textbf{Impact}\\
\hline
08:25:00 & BZ & Weekly Market Readout & 1-w Aug & - & Medium\\
15:00:00 & BZ & Weekly Trade Balance & 1-w Aug & - & Low\\
16:00:00 & US & Consumer Credit & June & 15,250b & Low\\
\end{tabular}
\end{center}

\textbf{Bottom Line}. A bit of global headwind and counterproductive flow of
news is likely to take dent in the early session.
\end{document}
